\newcommand*{\Signature}[2]{%
    \vspace{4ex}
    \par\noindent\makebox[1.0\textwidth]{\hrulefill}%
    \par\noindent\makebox[0.8\textwidth][l]{#1 (#2)}%
}%
\makeatletter
\thispagestyle{scrheadings}\markright{{Aalborg University Copenhagen} \hfill {M.Sc. Geoinformatics}}


\includegraphics[width=0.4\textwidth]{Chapters/aau_studentreport.pdf}
\label{fig:aau_studentreport}

\begin{minipage}[t][0.8\textheight]{0.45\textwidth}
  \textbf{Title:} FloodTool 2.0
  
  \textbf{Subtitle:} An online open source application for flood modelling

  \textbf{Theme:} Master thesis

  \textbf{Project period:} February, 2015 -- June, 2015

  %\textbf{Project group:}\\

  \textbf{Students: }
  \Signature{Ioannis Angelidis}{20132531}\\
  \Signature{David Nagy}{20140706}\\
  \Signature{Emil M. Rasmussen}{20130954}\\

  \textbf{Supervisor:} Thomas Balstrøm, Professor, Aalborg University
  Copenhagen\\

  \vfill
  %\textbf{Edition: }\\
  \textbf{Number of copies:} 2\\
  \textbf{Number of pages: 83} \\ %\pageref{SoLongAndThanksForAllTheFish}\\
  %\textbf{Enclosures:}\\
\textbf{Completed:} June 10, 2015\\
\end{minipage}
\hfill
\fbox{
  \begin{minipage}[t]{0.45\textwidth}
    \textbf{Synopsis:}

{\small Geographical Information Systems have become widespread during the past decades. This evolution has also encouraged the development of open source GIS software suites. The goal of this project is to demonstrate a way to combine various open source technologies in order to create an online application that offers the user the possibility to model a flood caused by sea water stowage.\\

We will begin by presenting the theory that is the foundation of this project. Then we will demonstrate in detail the way we have chosen to combine open source GIS with various programming languages, in order to create an online application. We will then present the outcome of that effort from a developer's point of view and what we think they will experience while using our application. We finish this report by discussing the choices we have made in order to achieve the goals we have set beforehand, but also what we think are the strengths and the weaknesses of such endeavor.}

  \end{minipage}
}
\makeatother