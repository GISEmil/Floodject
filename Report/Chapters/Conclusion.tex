\chapter{Conclusion}
The purpose of this project was to create an online application capable of performing simple flood modeling using open source technologies. In addition to that, we wanted to provide any interested parties with thorough documentation of the technologies used, how they are combined and how they interact with each other. We believe that we have achieved these tasks.\\

Working with open source technologies, has been an interesting experience for the development group, as they have never used open source technologies to such a degree as done during this project. During the process we encountered a series of obstacles which we either overcame or worked around, but surely this could be expected when basing an entire project on such software. 


\begin{itemize}
\item In spite of this, we have successfully combined PyWPS, GRASS, Flask and web development languages, into a fully operating application. Even though some changes have been incorporated in the tools we are using, we think that the tools successfully fulfill the expected requirements.
\item By simplifying the interface and the inputs required from the user, we believe that the application has a layer of complexity removed, that normally would discourage the usage of such an analysis.
\item Having thoroughly recorded the development phase, through the extensive implementation chapter, and also using the code repository GitHub, we believe that we have successfully created usable documentation. Keeping the open source aspect in mind, this documentation is freely available to any interested party.
All in all, we have fulfilled the goals specified in our problem statement, and created an application that fulfills our initial ambitions. We have thoroughly documented our ideas for future development and as well as problems encountered, in the hopes that it can benefit other GI professionals, and they can learn from our experiences. 