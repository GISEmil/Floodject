
\chapter{Perspectives}


In this part of the report we will present what we think might be good additions to the application, that can be on a short term. These additions will offer a more all-around flood modelling service. Additionally this section will contain items of technical and non-technical nature, but overall they include expansion to the application we have described in the sections before.

\paragraph{Barrier:} These functionalities include first and foremost the design of a barrier in the landscape. We presented beforehand the work we dedicated on that aspect of the application. What is missing from this extension is its connection to the core script. In addition, our plan includes the ability to define the height of the barrier. We believe that this extension can be easily added since the greatest part of the work needed is already completed. By adding the barrier feature, we will be considerably closer to a fully converted and working version of the older project but also offer to the user a more completed approach to the disaster management aspect of this project.

\paragraph{User input:} The deployed functionalities have a hard coded  maximum flood level input, which is set to 3 meters based on the danish flood history. However, for better usability, we would like to allow the user to interactively change which flood level extent they would like to simulate. That way we ensure our application is expanded to include the flood event the user wants to simulate. This would give a certain flexibility to the service and enable the user to adjust the service to their needs. 

\paragraph{Multi-Criteria Decision Analysis:} At the beginning of this project we have set up our goals, regarding what this project can be implemented for. Initially we wanted to create a useful application for disaster management purposes, including more thorough decision making features. For instance, after the flood modeling, the user could get some output based on existing infrastructure, population density as well as the land use of the investigated area. In order to show this information, it can be done by Multi Criteria Decision Analysis(MCDA).
MCDA can be a very valuable tool that can be applied for complex decision making based upon different flood protection scenarios. Setting up weights for the previously mentioned features, can be critical for flood management purposes. Likewise, establishing danger zones or high importance zones based on the area's importance, it helps users to think, re-think, adjust, test, and decide on a final scenario, which can mitigate the impact of a natural disaster. Using this feature, the users can benefit from  highlighting the critical areas. To give an example, we have created some possible output, which an MCDA function could provide to the user.

However to develop and deploy such functionality, it requires more user input, such as census data, network of infrastructure or land use maps. This input can also be invoked from different online services, such as  GeoDanmark. Of course it would need further adjustment and programming in order to work with our already existing service, but the time limitation of this project could not allow us to start this implementation.

\paragraph{Projections:} As mentioned previously, we have created the functionality that will enable GRASS to reproject any DEM into a new coordinate system, thus eliminating the geographical reference system limitation. To create an entirely dynamic application, with the capability of handling any DEM, will require a lot of work on the Flask back-end, to make sure that the DEM contains a projection and it gets reprojected to the right type.

The problem with using WGS84 is that data precision is lost when using very localized data. For this reason it would be critical to support the importing of any coordinate reference system, if this software would ever be launched for production.

\paragraph{User Accounts:} In order to offer the users with a more professional approach, we have considered in creating a feature of user accounts. The idea behind that comes from the fact that the application might have to deal with multiple users using it simultaneously. We have already stated that multiple uploads with the same name is an issue the application has, but using user accounts will change that. By creating folders based on a pre-designated user id, we can store each uploaded file to the respective folder and as a result organize the structure of the server in a more efficient manner.   

\paragraph{Service upgrade:} To begin with, the most realistic and easy to achieve goal is expanding the capabilities of our server. This mainly focuses on being able to handle larger input data in order to provide the user with more accurate results. This also includes increasing the processing power of the server, thus making the application even faster especially when large elevation models need processing. 
Promotion
Another goal is to make this application available to the public. Even though anyone can access the service even now, we would like to optimize and polish the overall outlook of it. Then making it known to the public that such application is available for free use would be a logical step in the direction that the group desires to take this project to.

\paragraph{Professional feedback:} From the beginning of the project, we have been discussing that it would be of great value for the project, if we could have feedback from other people on what do they think about the application. Quite often, especially when spending a great amount of time developing the application, we tend to make decisions or deal with issues with a very inflexible point of view. This "tunnel vision" might keep the group from discovering obvious limitations or faults while using the service. For that reason, we strongly believe that it would be essential to have other people using our application and providing us with information about the experiences they had through using it. That way, we will limit the drawbacks and locate faulty development of the application. The idea is to find as many people as possible, coming from different backgrounds, so that their viewpoint would be as diverse as possible. We would then examine their feedback and based on that, decide which changes we need to make in order to enhance  the application.

\paragraph{External developing:} By making sure that all the code and script are freely available to any interested party, we want to think that we encourage external feedback on the way we implemented the various aspects of our project. Asides from that, we hope that this process and its elements might reach out to other GI developers that might be willing to take our work further or even use it to develop another project. We are welcoming such opportunities and hope that we might receive feedback, advice or propositions from other specialists of the field, in order to expand, optimize and improve this current application or observe how parts of it can be implemented on other projects.

\paragraph{Wider applicability:} Now that the general shell of the project has been created, it would actually be relatively simple to replace the functionality with other geographic analysis. This would mean that the flooding is an arbitrary choice, and the shell quickly could be reused by us, or others, to create services that could perform analysis of various kinds.

Obviously, the learning-curve with other applications or methods could be an issue, as this has had a very specific focus. As mentioned it would be easy to apply this method to other phenomena, with the learning curve of understanding the actual problem being the issue, and not learning the technology. 