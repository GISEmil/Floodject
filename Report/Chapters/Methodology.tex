% Chapter 1

\chapter{Methodology} % Chapter title

\label{ch:theory} % For referencing the chapter elsewhere, use \autoref{ch:introduction} 

%----------------------------------------------------------------------------------------
METHODOLOGY
The works of this project officially started on the 2nd of February and the expected turn-in date was set on the 10th of June. This time-span provided the group with approximately four months to complete the development of this project and the report that accompanies it.
As far as the working schedule is concerned, we decided as a group that a fixed set of days each week, where the group would meet and work on the project is best suitable. The rest of the days, each member of the group would work individually on pre-assigned tasks that would be discussed among the group in the next available meeting. Taking all the above into account, we set a weekly schedule that was followed throughout the completion of this project (table XXX).
Monday 
Tuesday
Wednesday
Thursday
Friday
Saturday
Sunday
Meet
--
--
Meet
Meet
--
--

In an effort to maximize productivity and collaboration between group members, we decided to follow a classic software methodology which is called SCRUM. This choice was made due to the fact that this project includes significant amount of programming work and most of the members have experience working under this method. 
Being a sub-version of the Agile methodology, SCRUM basically adheres to a specific routine:
Which tasks have been progressed since yesterday?
Which tasks will be worked on tomorrow?
Are there any obstacles preventing tasks from completion?
Always keeping in mind this routine, three to four meeting were scheduled each week so that the group could discuss the direction and progress of the project. In addition, other means of connections (Dropbox, Google Drive) were established in order to maintain communication among the group members on the days that group meetings did not occur. During the meetings, group members could present propositions on how to enhance the project or seek assistance in the case where a task could not be completed.
Developing the software and writing the report did not occur in parallel fashion, but extensive notes have been kept and a log was created in order to document the important issues and queries that occurred during the development phase.
The main issue that occurred during the development phase of the project was that in some cases, code needed to be tested on the server to determine whether it performed without any errors. That fact, crippled the flexibility of the group on the occasions where we needed to test fixes for broken code. To be more specific, each time a fix was implemented, the server needed to be restarted in order to load the new script and test its new version. When restarting the server we had to make sure that no other group member was working on the server side, so a waiting gap existed between testing and fixing the code.


%----------------------------------------------------------------------------------------

\section{Agile}

%----------------------------------------------------------------------------------------

\section{Timeline}\label{sec:options}
\subsection*{Functions}


%----------------------------------------------------------------------------------------

\section{The phases of our project}\label{sec:custom}
