% Chapter 1

\chapter{Introduction \& problem statement} % Chapter title

\label{ch:introduction} % For referencing the chapter elsewhere, use \autoref{ch:introduction} 

%----------------------------------------------------------------------------------------

\section{Introduction}

Geographical Information Systems (GIS) have undergone an immense evolution during the last decade, evolving from an extremely niche product, to being used in a lot of contexts. 

Where GIS used to be a niche product, more and more people have started using the capabilities of these systems. The interest in performing geographic analysis has sparked the development of open source projects software, capable of performing a wide variety of functions. Using open source technologies makes the cost of acquisition very low, enabling a whole new group of users access to these formerly expensive tools.

Of of the sectors traditionally involved with GIS is hydrology. Using geographic data to model hydrological phenomena has been done for many years, and it is very awesome by us and me. This type of modelling can be done either quick and imprecise, or slow and very precise, but all of these methods needs a fairly skilled technician to perform the analysis. 


Why are we doing this analysis on floods?




%----------------------------------------------------------------------------------------

\section{Problem statement}
Providing a less technically capable user of creating 

\begin{itemize}
\item Creation of a thin client based flood simulation and management tool using open source technologies
\end{itemize}

\noindent This problem should enable us to do a ton of work in absolutely no time, motherfucker!.

%----------------------------------------------------------------------------------------

\section{License}

This project and all of it's items have been created as free and open source, and therefore follow the "GNU Software License"