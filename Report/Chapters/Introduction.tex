% Chapter 1

\chapter{Introduction \& problem statement} % Chapter title

\label{ch:introduction} % For referencing the chapter elsewhere, use \autoref{ch:introduction} 

%----------------------------------------------------------------------------------------

\section{Introduction}

Geographical Information Systems (GIS) are designed to manage and analyze geographic data. GIS used to be a niche product, but has over the last couple of decades become more widespread, and a plethora of products have become available. 
This evolution has also culminated in the development of open sourced GIS software. Because of their open source nature they have evolved to such a extent, that Open Source GIS software now can provide functionality comparable to what is available from the established commercial products. 
Using open source technologies as the back-end of your project, makes the cost of acquisition very low, enabling a whole new group of users access to these tools, and analysis methods. 
Whilst a lot of GIS functionality exist as libraries to Python, and as simple stand-alone applications, some all-inclusive Open Source GIS applications exist, amongst these the most well-known are QGIS and GRASS. 
One sector that historically has used GIS, is Hydrology. Performing geographical analysis on the movements, spread and aggregation of water in a landscape, is crucial to understanding the particular phenomena. 
Understanding hydrological movement is complicated, and often requires advanced modelling. 
Furthermore, “deciding” on what kind of inputs and outputs that can be provided by the user, will simplify the process extensively. “Hiding” the process from the user, makes the entire ordeal less complicated and more easily handled.


%----------------------------------------------------------------------------------------

\section{Problem statement}
This project is inspired by our own work with flood modelling using ArcGIS in a previous semester of studies. 
Combining a variety of Open Source technologies, it should be possible to create an application that will enable a user, who is not necessarily proficient with GIS, of performing a flooding analysis. As such our problem statement will be as follows:  

\begin{itemize}
\item Creation of a thin client based flood simulation and management tool using open source technologies
\end{itemize}

\noindent This problem should enable us to do a ton of work in absolutely no time, motherfucker!.

\section{Delimitation}
To base the project on the real world, the variables and constants used within the project will be based on usage within Denmark.   
There are several reasons for doing this. First of all, this is an area the project group has worked with before, and therefore all are comfortable with. This also creates a practical framework for the project – I.e we can create the product based on real-life values and data, instead of “making it up” as we go along. 
All of these can be changed at a later state, but we fill it makes sense to create them with a foundation in Denmark

%----------------------------------------------------------------------------------------

\section{License}

This project and all of it's items have been created as free and open source, and therefore follow the "GNU Software License".